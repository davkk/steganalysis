\subsection{Steganaliza statystyczna}

Proces ukrywania informacji w nośnikach cyfrowych zmienia ich właściwości statystyczne \cite{stat_stego_study}. Nośniki takie jak
obrazy, dźwięki czy wideo mają przewidywalne wzorce, które są zakłócane przez steganografię. Dzięki temu
stosując odpowiednie metody analizy statystycznej, można wykryć obecność ukrytych danych.

Najpopularniejszym nośnikiem steganograficznym wymienianym w literaturze są obrazy cyfrowe. Poniżej wymieniono
często stosowane techniki steganalizy statystycznej w kontekście obrazów cyfrowych \cite{Michaylov23012024}.
\begin{itemize}
    \item \textbf{Test chi-kwadrat}: Porównuje rozkład obserwowany z oczekiwanym, np. dla steganografii PVD \cite{chisquare}.
    \item \textbf{RS steganaliza}: Klasyfikuje grupy pikseli jako „regularne” lub „osobliwe” na podstawie zmian szumu po modyfikacji LSB \cite{RS}.
    \item \textbf{Raw Quick Pair (RQP)}: Analizuje pary kolorów w obrazach 24-bitowych, które różnią się w bitach LSB.
    \item \textbf{Analiza histogramu}: Wykrywa zmiany w rozkładzie wartości pikseli lub współczynników DCT.
\end{itemize}

\subsection{Analiza histogramów}

Steganaliza oparta na histogramach jest statystyczną metodą wykrywania ukrytych wiadomości w obrazach,
które zostały zmodyfikowane za pomocą technik takich jak steganografia w najmniej znaczących bitach (LSB).
Poniżej znajduje się opis implementacji metody z artykułu A Novel Steganalysis Method Based
on Histogram Analysis \cite{histogram_steg}.

\subsubsection{Algorytm}

Metoda opiera się na porównaniu histogramów obrazu wyjściowego (cover image) oraz podejrzanego obrazu
(suspicious image). Specyficzny wzór różnic histogramów pomaga wykrywać obrazy stego (obrazy zawierające
ukryte dane). Metodę zaprojektowano tak, aby była uniwersalna, czyli niezależna od zastosowanego algorytmu
steganograficznego.

Algorytm składa się z następujących kroków:
\begin{enumerate}
    \item Odczytanie podejrzanego i wyjściowego obrazu oraz zapisanie ich wartości intensywności pikseli.
    \item Obliczenie histogramów obu obrazów.
    \item Analiza różnic histogramów: Szukanie wartości sąsiednich, które mają tę samą wielkość, ale różne
    znaki.
    \item Zliczenie takich par i porównanie wyniku z ustalonym progiem.
    \item Na podstawie wyniku określenie, czy obraz podejrzany to obraz stego, czy normalny obraz.
\end{enumerate}

\begin{figure}[htbp]
    \centering
    \includegraphics[width=0.8\textwidth]{./img/hist_algorythm.png}
    \caption{Schemat działania algorytmu opartego na histogramach.}
    \label{fig:histogram_algorithm}
\end{figure}

\subsubsection{Implementacja}
Poniżej przedstawiono fragment programu z kodem zaimplementowanego algorytmu. Pełny kod dostępny jest w
repozytorium.

\begin{lstlisting}[language=Python, caption=Steganaliza oparta na histogramach w Pythonie]
import numpy as np
import matplotlib.pyplot as plt
from PIL import Image

def detect_stego_image(cover_image, suspicious_image, threshold):
    # Step 1: Wczytaj obrazy i zamień je na wartości intensywności (grayscale)
    cover = np.array(Image.open(cover_image).convert('L')) / 255.0
    suspicious = np.array(Image.open(suspicious_image).convert('L')) / 255.0

    # Step 2: Oblicz histogramy obu obrazów
    cover_histogram, _ = np.histogram(cover.flatten(), bins=256, range=(0, 1))
    suspicious_histogram, _ = np.histogram(suspicious.flatten(), bins=256, range=(0, 1))

    # Step 3: Wykres histogramów i różnice
    plt.bar(range(256), cover_histogram, alpha=0.5, label='Cover Image Histogram')
    plt.bar(range(256), suspicious_histogram, alpha=0.5, label='Suspicious Image Histogram')
    plt.legend()
    plt.title("Histogram Comparison")
    plt.xlabel("Pixel Intensity")
    plt.ylabel("Frequency")
    #plt.show()

    # Step 4: Różnice histogramów
    differences = suspicious_histogram - cover_histogram

    # Step 5: Zlicz różnice z przeciwnym znakiem i tą samą wartością (z tolerancją)
    counter = 0
    tolerance = 1e-6  # Tolerancja dla numerycznych błędów
    for i in range(len(differences) - 1):
        if abs(differences[i] + differences[i + 1]) < tolerance and abs(differences[i]) > 0:
            counter += 1

    # Step 6: Sprawdzenie wartości progowej licznika
    if counter > threshold:
        return True  # Stego Image
    else:
        return False  # Normal Image
\end{lstlisting}

Analizę wykonano dla próbek 20 obrazów stego i 20 obrazów cover pobranych z bazy BOSSBase. W celu sprawdzenia
wrażliwości algorytmu na modyfikacje obrazów nie będące działaniami steganograficznymi, pliki cover poddano
kompresji oraz wprowadzono losowy szum--w ten sposób otrzymano 20 próbek "compress".

Program sprawdził pary obrazów cover--stego oraz cover--compres i zwrócił wynik detekcji dla różnych
wartości progu detekcji.

\subsubsection{Wyniki}

Obrazy histogramów dla jednej z 20 par przedstawiono poniżej.
\begin{figure}[ht!]
    \centering
    \includegraphics[width=0.8\textwidth]{./img/compress_histogram.png}
    \caption{Porównanie histogramów obrazu cover i compress.}
    \label{fig:histogram_comparison}
\end{figure}

\pagebreak
\begin{figure}[ht!]
    \centering
    \includegraphics[width=0.8\textwidth]{./img/compress_diff_histogram.png}
    \caption{Histogram różnic między obrazami cover i compress.}
    \label{fig:histogram_diff}

\end{figure}

\begin{figure}[ht!]
    \centering
    \includegraphics[width=0.8\textwidth]{./img/cover_histogram.png}
    \caption{Porównanie histogramów obrazu cover i suspicious.}
    \label{fig:histogram_comparison}
\end{figure}

\begin{figure}[ht!]
    \centering
    \includegraphics[width=0.8\textwidth]{./img/diff_histogram.png}
    \caption{Histogram różnic między obrazami cover i suspicious.}
    \label{fig:histogram_diff}
\end{figure}

\pagebreak

\begin{table}[ht!]
    \centering
    \begin{tabular}{|p{3cm}|p{5cm}|p{5cm}|}
    \toprule
        \textbf{Próg detekcji} & \textbf{Poprawnie zidentyfikowane obrazy compress} & \textbf{Poprawnie zidentyfikowane obrazy stego} \\ \midrule
    5 & 19 / 20 & 18 / 20 \\
    6 & 19 / 20 & 16 / 20 \\
    4 & 18 / 20 & 19 / 20 \\ \bottomrule
    \end{tabular}
    \caption{wyniki klasyfikacji dla różnych progów detekcji.}
    \label{tab:classification_results}
    \end{table}

\pagebreak

\subsubsection{Dyskusja i wnioski}
Wyniki wyglądają bardzo dobrze. Program poprawnie zidentyfikował większość próbek. Sam algorytm był łatwy w
implementacji oraz intuicyjny. Jest również prosty i nie wymaga zaawansowanych obliczeń, dlatego analiza trwa
krótko.
Dobrze też wypada wrażliwość na szum--mimo wprowadzenia zmian do obrazów, które nie były ukrytymi
wiadomościami, algorytm poprawnie zidentyfikował je jako obrazy nie-stego.

Nie znany jest rodzaj steganografii, który został zastosowany do obrazów, więc nie można jednoznacznie określić
uniwersalności metody.

Minusem algorytmu jest to, że opiera się na porównaniu podejrzanego obrazu z oryginałem. W praktyce utrudnieniem
może być konieczność posiadania obrazu wzorcowego. Skuteczność zależy od ustawienia wartości progowej, co
również może być problematyczne.

\subsubsection{Podsumowanie}
Algorytm zaproponowany przez autorów artykułu okazał się skuteczny w wykrywaniu obrazów stego. Istnieje też
szansa rozwoju go na analizę steganografii nie tylko przestrzennej, ale też czasowej, co pozwoliłoby na rozszerzenie
analiz na inne nośniki cyfrowe, takie jak dźwięk czy wideo.
