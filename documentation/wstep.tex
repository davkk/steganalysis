\subsection{Steganografia i steganoanaliza}
\textbf{Steganografia} to sztuka ukrywania informacji. Jej celem jest przekazanie wiadomości w taki sposób, żeby 
pozostała niezauważona przez osoby postronne. Tym różni się od kryptografii - ta zabezpiecza informacje 
przed odczytaniem, natomiast sam komunikat może zostać zauważony, dopóki nie zostanie odkodowany. W 
steganografii sam fakt przekazywania informacji powinien zostać niezauważony przed podsłuchujących.
Sama technika steganografii powstała w starożytności, ale rozwój technologii znacząco wpłynął na rozwój tej 
dziedziny, dostarczając nowych metod ukrywania informacji. \cite{Johnson, khalid}

Nośnik, w którym zostanie ukryta informacja, może być właściwie dowolny i nazywany jest kontenerem. 
Najpopularniejszymi nośnikami w steganografii są dane multimedialne: obraz, pliki audio i wideo. Równie 
często wykorzystuje się protokoły IP i VoIP. Zdarzają się również techniki steganograficzne w tekstach, 
jednak są rzadziej stosowane z powodu różnego rodzaju ograniczeń.\\
Można ogólnie powiedzieć, że każdy nośnik danych jest potencjalnym kontenerem dla steganografii. 
Istnieją opisy wykorzystywania do tego plików wykonywalnych *.exe, wiadomości MMS, protokołów UDP czy nawet 
łańcuchów DNA. \cite{Kozie2014WspczesneTS}


Odpowiedzią na steganografię jest \textbf{steganoanaliza}, zajmująca się wykrywaniem ukrytych wiadomości. 
Steganaliza ma na celu identyfikację obecności ukrytych danych, a czasem nawet ekstrakcję tych danych, bez 
wiedzy o kluczu lub algorytmie użytym do osadzenia wiadomości. 
Steganalizę można podzielić na dwie główne kategorie: steganalizę specyficzną (target) oraz steganalizę 
uniwersalną (blind). Do pierwszej grupy należeć będą metody ukierunkowane na konkretne algorytmy 
steganograficzne. Druga grupa zbiera rozwiązania bardziej elastyczne, ale przez to najczęściej mniej dokładne. 

Wraz z rozwojem nowych technik steganograficznych, stale potrzebne są nowe i bardziej zaawansowane metody 
steganalizy -- steganaliza jest nieustannie rozwijającą się dziedziną, która odgrywa kluczową rolę w 
zapewnianiu bezpieczeństwa informacji i walce z cyberprzestępczością. \cite{stat_stego_study}


\subsection{Informacje nt. realizacji projektu}
Projekt polegał na opracowaniu teoretycznym i implementacji wybranych metod wykrywania ukrytych informacji 
w obrazach, plikach audio lub innych mediach.
W dalszej części dokumentacji zostały przedstawione wybrane techniki steganoanalizy wraz z opisem 
implementacji. Każdy członek zespołu opracował jedną metodę. Pełny kod źródłowy projektu znajduje się w 
repozytorium na platformie GitHub \cite{repo}.

